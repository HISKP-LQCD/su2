\documentclass[12pt, a4paper]{article}
\usepackage[utf8]{inputenc}

\title{HMC for staggered fermions}
\author{Simone Romiti}
\date{\today}

\usepackage{amsmath}

\usepackage[backend=biber, sorting=none, backref=true]{biblatex} % Bibliography package
\addbibresource{biblio.bib} % file with bibliographic items


\usepackage{hyperref}
\begin{document}

%\newcommand{\HMCref}{
%\href{https://www.sciencedirect.com/science/article/pii/0550321389903246}{[BKHMR]}.
%}

%\newcommand{\GLref}{
%\href{https://link.springer.com/book/10.1007/978-3-642-01850-3}{[GL]}.
%}




\maketitle

\section{Hybrid Monte Carlo for staggered fermions}

We want to include staggered fermions \cite{BITAR1989377} in the action of a $U(1)$ theory in $2+1$ dimensions.

\subsection{Steps of the implementation}

The HMC consists of the following steps:

\begin{enumerate}
    \item 
    \textbf{Heatbath}: 
    Generation of random momenta
    \item 
    \textbf{Molecular Dynamics (MD)}: 
    Evolution of the field configurations according to an effective Hamiltonian
    \item
    \textbf{Accept-Reject}: 
    Acceptance of rejection of the final configuration 
    according to some given probability distribution.
\end{enumerate}
%
The inclusion of fermions in the action means the presence of the fermionic determinant in the partition function. 
This can be evaluated as a gaussian integral over pseudo-fermionic degrees of freedom 
(spinor structure but with bosonic algebra):

\begin{equation}
\text{det} D = 
\int d \phi^{\dagger} d \phi 
\, \text{exp}(-\phi^{\dagger} D^{-1} \phi)
\end{equation}

For staggered fermions the Dirac operator is

\begin{equation}
D(x,y) = G(x,y) + m \delta_{x,y} = \\ 
=  \sum_{\mu} \frac{1}{2} \eta_{\mu}(x) 
    [U_{\mu}(x) \delta_{x+\mu, y} -
     U^{\dagger}_{\mu}(x-\mu) \delta_{x-\mu, y} ]
+ m \delta_{x, y}
\end{equation}

with $\eta_{\mu}(x) = \prod_{\nu<\mu} (-1)^{x_{\nu}}$.

The fermionic action is:

\begin{equation}
S_F = \sum_{x, y} \bar{\psi}(x) D(x,y) \psi(y)
\end{equation}

\subsection{Numerical implementation}

In order to generate pseudo-fermion fields with the heatbath method, is more convenient to evaluate instead (see eq. (10) of \cite{BITAR1989377}):

$$
\text{det} D^{\dagger} D = 
\int d \phi^{\dagger} d \phi 
\, \text{exp}(-\phi^{\dagger} (D^{\dagger} D)^{-1} \phi)
$$

interpreting the product of the determinants as coming from the degenerate $u$ and $d$ quarks. 
Note that staggered Dirac operator is anti-hermitian 
(see eq. (10.26) of \cite{gattringer2009quantum}), 
so that the above integral is equivalent to $( \text{det} D)^2$.

In practice, what we do is the following:

\begin{enumerate}
    \item
    Generate a gaussian vector $R$ of $N$ components 
    ($N$ is the number of lattice sites)
    \item
    Evaluate and store $\phi = D^{\dagger} R$.
    \item
    Evolve with the MD, keeping in mind that since:
    $$
    S_F = 
    \phi^{\dagger} (D^{\dagger} D)^{-1} \phi = 
    \phi^{\dagger} M^{-1} \phi
    $$
    
    we have:
    
    \begin{equation} \label{eq:dSF.generic}
    \delta S_F 
    = - \phi^{\dagger} M^{-1} \cdot \delta M \cdot M^{-1} \phi 
    = - \chi^{\dagger} \delta M \chi
    \end{equation}

    
    where $\chi = M^{-1} \cdot \phi$.
    \item
    After the MD trajectory we compute the new $R^{\dagger} R$ :
    $$
    \vec{R} = D \vec{\chi} = D \cdot ( (D^{\dagger} D)^{-1} \cdot \vec{\phi})
    $$
    
    where $\phi$ is the one computed at the beginning of the trajectory, 
    and $D$, $D^{\dagger}$ are evaluated from the new gauge configuration. 
    
    Note that $\chi$ is evaluated first, 
    so that we have to invert $(D^{\dagger} D)^{-1}$ and not $(D^{\dagger})^{-1}$ . 
    This is done in order to having to invert an hemitian matrix for which, for instance, 
    the Conjugate Gradient algorithm always converges.
    
    Note also that we don't find the inverse matrix explicitly, 
    but its application to a given vector: 
    if we want to compute $A^{-1} \vec{b}$, 
    we find the numerical solution $\vec{x}$ to $A \vec{x} = \vec{b}$.
\end{enumerate}

  

\subsection{Sparse matrix multiplication}

The matrices we deal with in the HMC are sparse. 
Instead of using a general-purpose library for their operations on vectors, 
we use their explicit expressions. 
The expression for the Dirac operator has been given above. 
We report here the expressions for $D^{\dagger} D$.

We can write: $D(x,y) = G(x,y) + m \delta_{x,y}$. 
This leads to:

\begin{equation}
\begin{aligned}
(D^{\dagger} D) (x,y) 
&= 
\sum_{z} G^{\dagger}(x,z) G(z,y)
+ m [ G^{\dagger}(x,y) + G(x,y) ]
+ m^2 \delta_{x,y} 
\\
&= M_1(x,y) + M_2(x,y) + M_3(x,y)
\end{aligned}
\end{equation}

The explicit expressions are:

\begin{enumerate}
\item
\begin{equation}
\begin{aligned}
  M_1(x,y)
  &= 
  \frac{1}{4} \sum_{\mu, \nu} \eta_\mu(x)
  [ 
\\
  &+ U_\mu^{\dagger}(x) U_\nu(x+\mu) 
  \eta_{\nu}(x+\mu) \delta_{x+\mu+\nu, y}
   \\
  &- U_{\mu}^{\dagger}(x) U_{\nu}^{\dagger}(x) 
  \eta_{\nu}(x+\mu) \delta_{x,y}
  \\
  &- U_{\mu}(x-\mu) U_{\nu}(x-\mu) 
  \eta_{\nu}(x-\mu) \delta_{x-\mu-\nu,y}
   \\
  &+ U_{\mu}(x-\mu) U_{\nu}^{\dagger}(x-\mu-\nu) 
  \eta_{\nu}(x-\mu) \delta_{x-\mu-\nu,y}
  ]
\end{aligned}
\end{equation}

\item
\begin{equation}
\begin{aligned}
    M_2(x,y) 
    = m \cdot \sum_{\mu} \frac{1}{2} \eta_{\mu}(x) 
    \{
      [ U_{\mu}(x) + U_{\mu}^{\dagger}(x) ] 
      \delta_{x+\mu, y} -
      [ U^{\dagger}_{\mu}(x-\mu) + U_{\mu}(x-\mu) ] 
      \delta_{x-\mu, y}
    \}  
\end{aligned}
\end{equation}

\item
\begin{equation}
  M_3(x,y) = m^2 \delta_{x,y}
\end{equation}

\end{enumerate}

\subsection{Derivative with respect to the gauge field}

The derivative with respect to $U_{\rho}(z)$ is defined as \cite{gattringer2009quantum}:
%
\begin{equation}
\frac{\partial }{\partial U_{\rho}(z)} f(U_\mu(x))
= \delta_{\mu \rho} \delta_{x, z} \sum_i T_i \frac{\partial}{\partial \omega_{\mu}^{i}} f(U)
=  \delta_{\mu \rho} \delta_{x, z}  \sum_i T_i \frac{\partial}{\partial \omega} 
f( e^{i \omega T_i} U)\big\vert_{\omega=0}
\end{equation}

In a $U(1)$ theory we have only one generator $T=1$, so:
\begin{equation}
\frac{\partial }{\partial U_{\rho}(z)} f(U_{\mu}(x))
=  \delta_{\mu \rho} \delta_{x, z}  \frac{\partial}{\partial \omega} f( e^{i \omega} U)\big\vert_{\omega=0}
\end{equation}

Therefore:
\begin{align}
\frac{\partial }{\partial U_{\rho}(z)} U_{\mu}(x) = 
i U_{\mu}(x) \delta_{\mu \rho} \delta_{x, z}
\\
\frac{\partial }{\partial U_{\rho}(z)} U^\dagger_{\mu}(x) = 
-i U^\dagger_{\mu}(x) \delta_{\mu \rho} \delta_{x, z}
\end{align}

It is also easy to verify that this derivative satisfy the Leibniz rule for the derivative of products.

Applying the above result we get the following results:

\begin{equation}
\frac{\partial D(x,y)}{\partial U_{\rho}(z)} 
=  \sum_{\mu} \frac{1}{2} \eta_{\mu}(x) 
    [i U_{\mu}(x) \delta_{x,z} \delta_{x+\mu, y}
     + i U^{\dagger}_{\mu}(x-\mu) \delta_{x-\mu,z} \delta_{x-\mu, y} ]
\end{equation}

\begin{equation}
\frac{\partial D^{\dagger}(x,y)}{\partial U_{\rho}(z)} 
=  \sum_{\mu} \frac{1}{2} \eta_{\mu}(x) 
    [-i U^{\dagger}_{\mu}(x) \delta_{x,z} \delta_{x+\mu, y}
     - i U_{\mu}(x-\mu) \delta_{x-\mu,z} \delta_{x-\mu, y} ]
\end{equation}

According to eq. (12) of \cite{BITAR1989377}, we evaluate the fermionic force $F_{\mu}(x)$ 
\footnote{
Is it correct to say that this is the quantity that we add to deriv(x, mu) in the code?
}
as:
%
\begin{equation}
F_{\mu}(x) = 
2 T
\left[ 
U_{\mu}(x) \frac{\partial S_F}{\partial U_{\mu}(x)} 
\right]
\end{equation}
%
Where $T$ is defined by eq. (5) of \cite{BITAR1989377}
\footnote{Is it correct that in our code this is implemented by get\_ deriv $<>$() ?}
%
With the notation of eq. \eqref{eq:dSF.generic} we write 
(compare with eq. (8.44) of \cite{gattringer2009quantum}):
%
\begin{equation}
\frac{\partial S_F}{\partial U_{\mu}(x)}
= 
- \chi^{\dagger}
\left( 
\frac{\partial D^{\dagger}}{\partial U_{\mu}(x)} D
+
D^{\dagger} \frac{\partial D}{\partial U_{\mu}(x)}
\right) 
\chi
\end{equation}

We observe that since
%
\begin{equation}
\frac{\partial D^{\dagger}(x,y)}{\partial U_{\rho}(z)} =
\left(
\frac{\partial D(x,y)}{\partial U_{\rho}(z)}
\right)^{\dagger}
\quad ,
\end{equation}
%
we have:
%
\begin{equation}
\frac{\partial S_F}{\partial U_{\mu}(x)}
= 
- \chi^{\dagger}
\left( 
\frac{\partial D^{\dagger}}{\partial U_{\mu}(x)} 
D
+
\left(
\frac{\partial D^{\dagger}}{\partial U_{\mu}(x)}
D
\right)^\dagger
\right) 
\chi
\end{equation}
%
Calling $Q = \frac{\partial D^{\dagger}}{\partial U_{\mu}(x)}
D$, the above equation leads to:
%
\begin{equation}
\frac{\partial S_F}{\partial U_{\mu}(x)}
= - \chi^{\dagger}( Q + Q^\dagger ) \chi 
= - 2 \operatorname{Re} \left( \chi^{\dagger} Q  \chi \right)
= - 2 \operatorname{Re} \left( \chi^{\dagger}  \frac{\partial D^{\dagger}}{\partial U_{\mu}(x)} D  \chi \right)
\end{equation}
%








\printbibliography


\end{document}
