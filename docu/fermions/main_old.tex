\documentclass[12pt, a4paper]{article}
\usepackage[utf8]{inputenc}

\title{U1}
\author{Simone Romiti}
\date{\today}

\usepackage{amsmath}

\usepackage{hyperref}
\begin{document}

\newcommand{\HMCref}{
\href{https://www.sciencedirect.com/science/article/pii/0550321389903246}{[BKHMR]}.
}

\newcommand{\GLref}{
\href{https://link.springer.com/book/10.1007/978-3-642-01850-3}{[GL]}.
}




\maketitle

\section{Hybrid Monte Carlo for staggered fermions}

We want to include fermions in the action of a $U(1)$ theory in $2+1$ dimensions. We use the staggered discretization for fermions as in \HMCref{}.

\subsection{Steps of the implementation}

The HMC consists of the following steps:

\begin{enumerate}
    \item 
    \textbf{Heatbath}: Generation of random momenta
    \item 
    \textbf{Molecular Dynamics (MD)}: evolution of the field configurations according to an effective Hamiltonian
    \item
    \textbf{Accept-Reject}: Acceptance of rejection of the final configuration according to some given probability distribution.
\end{enumerate}
%
The inclusion of fermions in the action means the presence of the fermionic determinant in the partition function. This can be evaluated as a gaussian integral over pseudo-fermionic degrees of freedom (spinor structure but with bosonic algebra):

\begin{equation}
\text{det} D = 
\int d \phi^{\dagger} d \phi 
\, \text{exp}(-\phi^{\dagger} D^{-1} \phi)
\end{equation}

For staggered fermions the Dirac operator is

$$
D(x,y) = G(x,y) + m \delta_{x,y} = \\ 
=  \sum_{\mu} \frac{1}{2} \eta_{\mu}(x) 
    [U_{\mu}(x) \delta_{x+\mu, y} -
     U^{\dagger}_{\mu}(x-\mu) \delta_{x-\mu, y} ]
+ m \delta_{x, y}
$$

with $\eta_{\mu}(x) = \prod_{\nu<\mu} (-1)^{x_{\nu}}$.

The fermionic action is:

$$
S_F = \sum_{x, y} \bar{\chi}(x) D(x,y) \chi(y)
$$

\subsection{Numerical implementation}

In order to generate pseudo-fermion fields with the heatbath method, is more convenient to evaluate instead (see eq. (10) of \HMCref{}):

$$
\text{det} D^{\dagger} D = 
\int d \phi^{\dagger} d \phi 
\, \text{exp}(-\phi^{\dagger} (D^{\dagger} D)^{-1} \phi)
$$

interpreting the product of the determinants as coming from the degenerate $u$ and $d$ quarks. Note that staggered Dirac operator is anti-hermitian (see eq. (10.26) of \GLref{}), so that the above integral is equivalent to $( \text{det} D)^2$.

In practice, what we do is the following:

\begin{enumerate}
    \item
    Generate a gaussian vector $R$ of $N$ components ($N$ is the number of lattice sites)
    \item
    Evaluate and store $\phi = D^{\dagger} R$.
    \item
    Evolve with the MD, keeping in mind that since:
    $$
    S_F = 
    \phi^{\dagger} (D^{\dagger} D)^{-1} \phi = 
    \phi^{\dagger} M^{-1} \phi
    $$
    
    we have:
    
    $$
    \delta S_F 
    = \phi^{\dagger} M^{-1} \cdot \delta M \cdot M^{-1} \phi 
    = \chi^{\dagger} \delta M \chi
    $$
    
    where $\chi = M^{-1} \cdot \phi$.
    \item
    After the MD trajectory we compute the new $R^{\dagger} R$ :
    $$
    \vec{R} = D \vec{\chi} = D \cdot ( (D^{\dagger} D)^{-1} \cdot \vec{\phi})
    $$
    
    where $\phi$ is the one computed at the beginning of the trajectory, and $D$, $D^{\dagger}$ are evaluated from the new gauge configuration. Note that $\chi$ is evaluated first, so that we have to invert $(D^{\dagger} D)^{-1}$ and not $(D^{\dagger})^{-1}$ . This is done in order to having to invert an hemitian matrix for which, for instance, the Conjugate Gradient algorithm always converges.
    Note also that we don't find the inverse matrix explicitly, but its application to a given vector: if we want to compute $A^{-1} \vec{b}$, we find the numerical solution $\vec{x}$ to $A \vec{x} = \vec{b}$.
\end{enumerate}

  

\subsection{Sparse matrix multiplication}

The matrices we deal with in the HMC are sparse. Instead of using a general-purpose library for their operations on vectors, we use their explicit expressions. The expression for the Dirac operator has been given above. We report here the expressions for $D^{\dagger} D$.

We can write: $D(x,y) = G(x,y) + m \delta_{x,y}$. This leads to:

$$
(D^{\dagger} D) (x,y) = 
\sum_{z} G^{\dagger}(x,z) G(z,y)
+ m [ G^{\dagger}(x,y) + G(x,y) ]
+ m^2 \delta_{x,y} \\
= M_1 + M_2 + M_3
$$

The explicit expressions are:

\begin{enumerate}
\item
\begin{equation}
\begin{aligned}
  M_1 = 
  \frac{1}{4} \sum_{\mu, \nu} \eta_\mu(x)
  [ 
  + U_\mu^{\dagger}(x) U_\nu(x+\mu) 
  \eta_{\nu}(x+\mu) \delta_{x+2\mu, y}
%   \\
  - U_{\mu}^{\dagger}(x) U_{\nu}^{\dagger}(x) 
  \eta_{\nu}(x+\mu) \delta_{x,y}
  \\
  - U_{\mu}(x-\mu) U_{\nu}(x-\mu) 
  \eta_{\nu}(x-\mu) \delta_{x,y}
%   \\
  + U_{\mu}(x-\mu) U_{\nu}^{\dagger}(x-2\mu) 
  \eta_{\nu}(x-\mu) \delta_{x-2\mu,y}
  ]
\end{aligned}
\end{equation}

\item
\begin{equation}
\begin{aligned}
    M_2 = m \cdot \sum_{\mu} \frac{1}{2} \eta_{\mu}(x) 
    \{
      [ U_{\mu}(x) + U_{\mu}^{\dagger}(x) ] 
      \delta_{x+\mu, y} -
      [ U^{\dagger}_{\mu}(x-\mu) + U_{\mu}(x-\mu) ] 
      \delta_{x-\mu, y}
    \}  
\end{aligned}
\end{equation}

\item  
\begin{equation}
  M_3 = m^2 \delta_{x,y}
\end{equation}

\end{enumerate}

\subsection{Derivative with respect to the gauge field}

Deriving with respect to $U_{\rho}(z)$ we obtain:

\begin{enumerate}
    \item 
For $M_1$ we find:
  
\begin{equation}
\begin{aligned}
  &\frac{\delta M_1(x,y)}{\delta U_{\rho}(z)} = 
  \frac{1}{4} \sum_{\mu} 
  \\
  &+ 
  \left[ 
  \eta_{\rho}(x) \eta_{\mu}(x+\mu)
  \frac{\delta U^{\dagger}_{\rho}(z)}{\delta U_{\rho}(z)} 
  \delta_{x,z} U_\mu(x+\mu)
  + 
  \eta_{\mu}(x) \eta_{\rho}(x+\mu)
  U^{\dagger}_{\mu}(x)
  \delta_{x+\mu, z} 
  \right]
  \delta_{x+2\mu, y} 
  \\
  &-\left[
  \eta_{\rho}(x) \eta_{\mu}(x+\mu) 
  \frac{\delta U^{\dagger}_{\rho}(z)}{\delta U_{\rho}(z)} 
  \delta_{x,z} U^{\dagger}_\mu(x)
  + 
  \eta_{\mu}(x) \eta_{\rho}(x+\mu) 
  U^{\dagger}_\mu(x)
  \frac{\delta U^{\dagger}_{\rho}(z)}{\delta U_{\rho}(z)} 
  \delta_{x,z}
  \right]
  \delta_{x,y}
  \\
  &-\left[
  \eta_{\rho}(x) \eta_{\mu}(x-\mu) 
  \delta_{x,z} U^{\dagger}_\mu(x-\mu)
  +
  \eta_{\mu}(x) \eta_{\rho}(x-\mu)
  U^{\dagger}_\mu(x-\mu)
  \delta_{x,z}
  \right]
  \delta_{x,y}
  \\
  &+\left[ 
  \eta_{\rho}(x) \eta_{\mu}(x-\mu)  
  \delta_{x-\mu,z} U^{\dagger}_\mu(x-2\mu)
  +\eta_{\mu}(x) \eta_{\rho}(x-\mu) 
  U_{\mu}(x-\mu)
  \frac{\delta U^{\dagger}_{\rho}(z)}{\delta U_{\rho}(z)}
  \delta_{x-2\mu, z} 
  \right]
  \delta_{x-2\mu, y}
\end{aligned}
\end{equation}

\item
We have:
  
\begin{equation}
    \sum_{x, y} 
    \phi^{\dagger}(x) 
    \frac{\delta M_2(x, y)}{\delta U_{\rho}(z)} 
    \phi(y)
    \\ =
    m \frac{1}{2} \eta_\rho(z) 
    \left(
    1 + \frac{\delta U^{\dagger}_{\rho}(z)}{\delta U_{\rho}(z)}
    \right)
    \left[
    \phi^{\dagger}(z) \phi(z+\mu)
    -\phi^{\dagger}(z+\mu) \phi(z)
    \right]
\end{equation}


In fact:
\begin{enumerate}
    \item 
    \begin{equation}
    %\frac{\delta D(x, y)}{\delta U_{\rho}(z)} =
    \frac{\delta G(x, y)}{\delta U_{\rho}(z)} =
    \frac{1}{2} \eta_\rho(z) 
    \left[ 
    \delta_{x,z} \delta_{y, z+\mu} -
    \delta_{x,z+\mu} \delta_{y, z} 
    \frac{\delta U^{\dagger}_{\rho}(z)}{\delta U_{\rho}(z)} 
    \right]
    \end{equation}
    
    implying:
    
    \begin{equation}
    \sum_{x, y} 
    \phi^{\dagger}(x) 
    \frac{\delta G(x, y)}{\delta U_{\rho}(z)} 
    \phi(y)
    \\=
    \frac{1}{2} \eta_\rho(z) 
    \left[
    \phi^{\dagger}(z) \phi(z+\mu) 
    -\phi^{\dagger}(z+\mu) 
    \frac{\delta U^{\dagger}_{\rho}(z)}{\delta U_{\rho}(z)} \phi(z)
    \right]
    \end{equation}
    %
    \item
    %
    \begin{equation}
    \frac{\delta G^{\dagger}(x, y)}{\delta U_{\rho}(z)} =
    \frac{1}{2} \eta_\rho(z) 
    \left[
    \frac{\delta U^{\dagger}_{\rho}(z)}{\delta U_{\rho}(z)}
    \delta_{x,z} \delta_{y, z+\mu} -
    \delta_{x,z+\mu} \delta_{y, z} 
    \right]
    \end{equation}
    
    implying:
    
    \begin{equation}
    \sum_{x, y} 
    \phi^{\dagger}(x) 
    \frac{\delta G^{\dagger}(x, y)}{\delta U_{\rho}(z)} 
    \phi(y)
    \\ =
    \frac{1}{2} \eta_\rho(z) 
    \left[
    \phi^{\dagger}(z) \phi(z+\mu) 
    \frac{\delta U^{\dagger}_{\rho}(z)}{\delta U_{\rho}(z)}
    -\phi^{\dagger}(z+\mu) \phi(z)
    \right]
    \end{equation}
%
\item
%
And of course:
\begin{equation}
\frac{\delta M_3}{\delta U_{\rho}(z)} = 0
\end{equation}
\end{enumerate}

\end{enumerate}











\end{document}
